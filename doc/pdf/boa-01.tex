\documentclass[12pt]{article}

\usepackage{amssymb,amsmath,xspace}
\usepackage[margin=3cm]{geometry}
\usepackage{amsthm}
%\usepackage{epigraph}


%%%%%%%%%%%%%%%%%%%%%%%%%%%%%%%%%%%%%%%%%%%%%%%%%%%%%%%%%%%%%%%%%%%%%%%%%%%%
%%%%%%%%%%%%%%%%%%%%%%%%%%%%%%%%%%%%%%%%%%%%%%%%%%%%%%%%%%%%%%%%%%%%%%%%%%%%
% THE DOCUMENT 

\begin{document}

\title{On choosing a suitable score function for the Bayesian Ontology Alignment tool}
\date{December 27, 2010}
\author{Vladimir Menkov}

\maketitle

\begin{abstract}
This paper describes techniques for computing confusion matrix
coefficients in the Bayesian Ontology Alignment tool (BOA). A
particular attention is given to formulas that have an
easy-to-understand meaning in the case of all cells of the data
sources containing values from some small set, but, at the same time,
can be expressed in terms of probability estimates given by a ``black
box'' PLRM model, and thus generalized to the case of an arbitrary
probability-generating model.
\end{abstract}

\section{The Ontology Alignment Task}

Consider the following problem. We are a given a table with $N_1$ rows
and $M_1$ columns ($C_1$, $C_2$, \dots, $C_{M_1}$), representing a
sampling of data from ``Data Source'' DS1, i.e. something like a SQL
database table. Each cell of the table contains an object from some
set $\cal{V}$, whose exact nature we can abstract from. Each row of
the table represent a structured data ``record'' of some kind, for
example a news article, and each cell of the table correspond to a
particular data element of the record's data item - e.g. the text
strings containing the title, the main text, the name of the first
author, the name of the second author (if any), the date, the place,
etc. of the article. 

There is also another table, with $N_1$ rows and $M_1$ columns
($C'_1$, $C'_2$, \dots, $C'_{M_2}$), which represents a sampling of
data from another ``Data Source'' DS2. The cell values in this table
also belong to the set $\cal{V}$.

We use the notation $z_{ij}$ and $z'_{ij}$ for the values in the cell
in row $i$, column $j$, of DS1 or DS2 respectively.

While the second table is quite different from the first, it is
thought that the data in the two table correspond to the real-world
objects of the same, or related types; it is also thought that the the
records' data are divided in somewhat similar way into columns for the
representation in the two table, even though the names of the columns
in the two tables are different. Our task consists in ``aligning the
ontologies'', i.e. figuring out which columns of the second table
correspond to which columns of the first table. The results should be
expressed in terms of a {\bf ``confusion matrix''}, which will contain a
number for each pair ($C_i$, $C'_j$). Rows of the confusion matrix
will correspond to columns of DS2, and columns of the confusion
matrix, to columns of DS1.

We may consider the set of possible cell values $\cal{V}$ to be a
subset of some finite-dimensional linear space $\cal{U}$, although in
a special case we are going to consider that fact will be only of a
limited importance.

\section{Algorithm overview}
\label{sec:overview}

The family of ontology alignment algorithms we'll be considering will
be based on underlying algorithms that can in some way classifiy
elements of the set $\cal{V}$ (i.e. set contents) with respect to
their propensity to be found in various columns of DS1. Based on the
numbers for individual cell contents of DS2, we then try to compute
plausible ``confusion matrix'' coefficients linking the column of DS2
with those of DS1.

The overall algorithmic framework can be described as follows:

\begin{enumerate}
\item[1] Consider the set of $M_1$ fields of DS1 as a single
  discrimination (set of labels) with $M_1$ classes (one per field)

\item[2] Create $M_1 \cdot N_1$ training examples, each example being
  the content of one field of one record from DS1, and carrying the
  class label equal to the name of the field in question. (When the data are
  represented with records as rows and fields as columns, each
  "example" introduced at this step will correspond to the content of
  one cell of this table).

\item[3] Tokenize etc. each "example" somehow, converting it into a
  vector in some linear space (a feature vector)

\item[4] Use some kind of Bayesian regression learning algorithm, such
  as one of those implemented by BOXER toolkit learner, on that set of $M_1
  \cdot N_1$ ``examples'', to come up with a classifier model that
  probabilistically assigns each ``example'' to a ``class'' (i.e., a
  field).  While predictions for individual examples may be of poor
  quality (e.g., when several columns are all filled with "Yes" and
  "No" values), this can, for example, capture the fact that the
  "Yes"/"No" ratio is higher in some columns than in others.

\item[5] Create $M_2\cdot N_2$ test examples, from the "cells" of DS2, in a
similar way to Step 2. Convert each example to a feature vector, as in
Step 3.

\item[6] Apply the classifier model obtained in Step 4 to these $M_2
  \cdot N_2$ test examples (cells of DS2). For each one, an array of
  $M_1$ probabilities  (summing to 1.0) will be thus computed,
  describing the likelihood of this particular cell belonging to
  something similar to each column of the B.

\item[7] For each column $C'_j$ of the DS2 we now have $N_2$ arrays
  (one for each cell) of $M_2$ probabilities each. We then compute
  each confusion matrix value $f_{ij}$, describing the level of
  ``connectedness'' of $C'_j$ with DS1's column $C_i$, based on the
  $N_2$ values obtained in Step 6 for the cells of $C'_j$. The process
  whereby this aggregate value $f_{ji}$ is not specified at this time;
  thus there is, generally, no guarantee that $\sum_i {f_{ij}}=1$, and
  $f_{ij}$ can be considered as a proper probability
  $P(C_i|C'_j)$. 

\item[8] While the values computed in Step 7 may or may not be
  interpreted as probabilities, the assumption is that, for a given
  $C'_j$, a greater value corresponds to a greater degree of
  connectedness. We can thus pick such $i$ that $f_{ij} > f_{kj}$ for
  any $k\ne i$, and say that $C'_j$ has the closest association with
  $C_i$. In other words, we will call $C_i$ the ``best match'' for
  $C'_j$.

\end{enumerate}

\section{Some properties of the Bayesian model}

Let us assume that the learning algorithm used in Step 4 is efficient
enough, and is able to construct a PLRM model very close, in terms of
the log-likelyhood, to the optimal one for the circumstances. What can
be said about this model? Obviusly, in general the properties of the
model, and in particular the probability values $P(C_i|v)$ it will
ascribe to the assignment of various elements of $\cal{V}$ to various
columns of DS1, will depend on how the elements of $\cal{V}$ have been
converted to feature vectors. However, under a certain - often not
unjustified - assumptions, the particular feature selection and the
particular linear regression algorithm won't matter much. 

{\bf Assumption 1.} The elements of $\cal{V_1} \subset \cal{V}$, the
set of all values of cells of DS1, have been converted to linearly
independent vectors.

The above assumption holds e.g. if each distinct cell value has a
particular ``shibboleth'' - a word that occurs in no other cell whose
entire text is different from this cell's text. 

This is the case, for example, if each cell contains a single word, or
is empty; our feature space consists of all words occurring in the
cells, plus the special ``empty'' token (thanks to Paul for this
idea!); and each cell's content is converted to a vector with a single
co-ordinate set.

Under Assumption 1, the following holds about the optimal Bayesian model 
one can build:

Let $\alpha_i(v)$ be the percentage of cells in column $C_i$ that
contain the value $v$. (Thus, $\sum_{v\in \cal{V}} \alpha_i(v) =
  1$). Then the Bayesian probability of assigning the value $v$ to
  column $C_i$ is
\begin{equation}
\label{eq:bp}
P(C_i|v) = \frac{\alpha_i(v)}{\sum_{j=1,\ldots,M_1} \alpha_j(v)}.
\end{equation}
In other words, the probability of assigning a given value $v$ to a
  particular column $C_i$ is proportional to the share of the cells
  with $v$ in the entire table that are located in column $C_i$.

\section{Formulas for aggregating probabilities}
\label{sec:agg}

{\bf Assumption 1.} All cell values found in DS2 are also found somewhere in DS1. 

In other words, $\cal{V}_2 \subset \cal{V}_1$, where $\cal{V}_2$ is
the set of values of cells of DS2.

The above is generally ''not'' the case - in when it is not the case,
tokenization and conversion from $\cal{V}$ to the feature space {\em
  do} matter; but in order to analyze certain simpler situations we
will work under this assumption. The situation we have in mind is that
of very limited vocabulary, when most table columns contain
essentially the same values, just in different proportions.

Similarly to the definition of $\alpha_I(v)$, let us define
$\gamma_j(v)$ as the proportion of the cells in DS2's column $C'_j$
that contain the value of $v$. Thus,  $\sum_{v\in \cal{V}} \gamma_i(v)=1$.

We will now consider how, in Step 7, individual probabilities for
cells within a column can be aggregated into the confusion matrix
elements for the column.

{\bf Arithmetic mean}
One way to compute the confusion matrix value $f_{ij}$ would be by averaging $P(C_i|v)$ for all cells of the column $C'_j$, which would give us
\begin{equation}
\label{eq:ag-ar}
f_{ij}^{\rm method 1} = R(C_i,C'_j) \equiv \frac{1}{N_2} \sum_{k=1}^{N_2} P(C_i|z'_{kj}) =
\sum_{\cal{V}} \gamma_j(v) P(C_i|v) 
\end{equation}
An advantage of this method is that $\sum_{i=1\ldots,M_1} f_{ij} = 1$, and the values can be easily interpreted as probabilities. Moreover, if Assumption 1 holds, the confusion matrix would be symmetric when the two data sources are identical (i.e, $\gamma_i(v)=\alpha_i(v)$ for all $i$), since in this case 
$$
f_{ij}^{\rm method 1} =\frac{\sum_{\cal{V}} \alpha_i(v)  \alpha_j(v)}{\sum_{j=1,\ldots,M_1} \alpha_j(v)}.
$$

{\bf Geometric  mean}
Another method is to use the geometric mean instead of the arithmetic
mean, computing the confusion matrix coefficients as 
\begin{equation}
\label{eq:ag-geo}
f_{ij}^{\rm method 2} = 
 \left(\prod_{k=1}^{N_2} P(C_i|z'_{kj})\right)^{ \frac{1}{N_2}}=
\prod_{\cal{V}}  P(C_i|v)^{\gamma_j(v)}.
\end{equation}
Multiplying probabilities can, of course, be interpreted as adding their logarithms.

Since the geometric mean of non-negative numbers of always no greater than the arithmetic mean of non-negative numbers, we know that $f_{ij}^{\rm method 2} \le f_{ij}^{\rm method 1}$ for all pairs of columns, and the values for a given $j$ do not sum to 1 anymore.

Note also that the average geometric is zero when any of the
participant columns is zero. Thus if even a single cell of the column
$C'_j$ contains a value that is {\em not} found in the column $C_i$ of
DS1, then $f_{ij}^{\rm method 2}$ will be 0. If the cells of column
$C'_j$ mixes values in a way not seen in {\em any} column of DS1 ---
that is, for every $i \in {1, \ldots, M_1}$ there is some value $v$
found in $C'_j$ but not found in $C_i$ --- then {\em every} coefficient 
$f_{ij}^{\rm method 2}$ for column $C'_j$ will be zero; that is, our method will say that $C'_j$ is not similar at all to any column of DS1.

{\bf Cosine similarity.}  Both of the methods above are not
particularly good when what we want to distinguish are columns that
are composed of the same values and are only different by the
proportions of those values. What we'd like to have is a confusion
matrix whose element $f_{ij}$ is maximized whenever the vector of
$\vec{\gamma_j}$, whose components are the relative frequencies
$\{\gamma_j(v)\}_{v\in \cal{V}}$ of various values in $C'_j$ is the
same as the vector $\vec{\alpha_i}$ of relative frequencies of various
values in $C_i$. A natural approach here would be a weighted cosine formula,
$$
f_{ij}^{\rm cosine \ method} = \frac{\sum_v{ \alpha_i(v) \gamma_j(v) \phi(v)}}{
\left(\sum_{v\in \cal{V}}{\alpha_i(v)^2  \phi(v)}\right)^{1/2}
\left(\sum_{v\in \cal{V}}{\gamma_j(v)^2  \phi(v)}\right)^{1/2}
},
$$
with some reasonable term-weight function $\phi(v)$.

Ideally, we would like the formula for $f_{ij}^{\rm cosine \ method}$
to be computable purely on the basis of probabilities for the cells,
$P(C_i|z'_{kj})$, and without explicitly using the values of
$\alpha_i$ and $\gamma_j$. This would allow us to naturally expand the
use of the formula even on the situation when Assumption 1 does not
entirely hold.

Considering the formula for the Bayesian probability (\ref{eq:bp}), we
note that the following weight would work very well for our purpose:
$$
\phi(v) = \frac{1}{\sum_{j=1,\ldots,M_1} \alpha_j(v)}.
$$ This kind of weight is readily interpreted as the inverse of the
overall frequency of a particular cell value in the entire table
DS1. Using it gives as the following scoring formula:
\begin{eqnarray}
\label{eq:ag-cos2}
f_{ij}^{\rm cosine \ method} &=& \frac{\sum_v{ P(C_i|v) \gamma_j(v)}}{
\left(\sum_{v\in \cal{V}}{P(C_i|v)\alpha_i(v)}\right)^{1/2}
\left(\sum_{v\in \cal{V}}{\gamma_j(v)^2  \phi(v)}\right)^{1/2}} \\
&=& \frac{ R(C_i,C'_j) }{ R(C_i,C_i)^{1/2} 
\left(\sum_{v\in \cal{V}}{\gamma_j(v)^2  \phi(v)}\right)^{1/2}}.
\end{eqnarray}

The values $R(C_i,C'_j)$ and $R(C_i,C_i)$ are simply the arithmetic
means introduced in eq. (\ref{eq:ag-ar}) above, and computable without
reference to Assumption 1. Unfortunately, the last factor,
$\|\vec{\gamma_j}\|_\phi=\left(\sum_{v\in \cal{V}}{\gamma_j(v)^2 \phi(v)}\right)^{1/2}$ is
  {\em not} computable without reference to Assumption 1. However,
  this $\|\vec{\gamma_j}\|_\phi$ is a factor common to $f_{ij}$ for all $i$ for a given
  $j$. Thus if we simply want to rank the columns of DS1 according to
  their ``similarity'' to $C'_j$, we can simply compute the ratios
\begin{equation}
\label{eq:ag-cos3}
s_{ij} = 
 \frac{ R(C_i,C'_j) }{ R(C_i,C_i)^{1/2} }
\end{equation}
When the matrix of these ratios $s_{ij}$ is reported as the confusion
matrix, one can compare values within the same row of this matrix, but
not between rows.

\section{Unequal-size samples from different columns}
The preceding discussion was carried in the assumption that we have
data from the equal number ($N_1$) of cells from each column of
DS1. Similarly, we had $N_2$ cells from each column of DS2.

What if we have differently-sized samples from different columns of a
data source? This situation can result from the sampling process, or
this can stem from the decision to {\em ignore} empty cells of the
data source, instead of choosing to treat them as legitimate cells
containing a value (an empty string).

Now, our (sample of) of DS1's column $C_i$ will consist of $n_i$
cells; $N_1$ will be understood as $\max_i n_i$. Similarly, column
$C'_j$ of DS2 will have $n'_j$ cells, and $N_2=\max_j n'_j$.

How will this situation affect the formulas for the confusion matrix
elements proposed above? 

It appears that the formulas for the {\bf arithmetic mean} (\ref{eq:ag-ar})
and {\bf geometric mean} (\ref{eq:ag-geo}) of the probabilities don't need
to be modified. Note that, by definition of Bayesian probabilities, if
a particular (sampled) column $C_l$ of of DS1 has exactly the same
composition of values of the column $C_i$, but we have fewer cells in
our samples from $C_l$ than from $C_i$ (i.e., $n_l < n_i$, than all
probabilities $P(C_l|V)$ will be proportioablly smaller than
$P(C_i|V)$:
$$
\frac{P(C_l|V)}{P(C_i|V)} = \frac{n_l}{n_i}.
$$ The arithmetic and geometric averages $f_{lj}$ will, too, be
proportionally smaller than $f_{ij}$, i.e. $f_{lj}/f_{ij}=n_l/n_i$.

For extending the {\bf weighted cosine similarity} formula
(\ref{eq:ag-cos2}), (\ref{eq:ag-cos3}) to the case of differently-sized
columns, we will use a different approach: namely, we will continue
defining $\alpha_i(V)$ as the ratio of the cells whose value is $V$
among the $n_i$ cells of column $C_i$. The values of $\gamma_j(V)$
will be defined similarly with respect to $C'_j$. We will still want
to define the cosine similarity $f_{ij}^{\rm cosine \ method}$ as the
cosine of the angle (in the weighted-dot-product space) between the
vectors $\vec{\alpha_i}$ and $\vec{\gamma_j}$; that is, if our samples
of columns $C_i$ and $C_l$ have exactly the same composition, even
though $n_i \ne n_l$, we'll want $f_{lj}^{\rm cosine \ method} =
f_{ij}^{\rm cosine \ method}$ for any $C'_j$.

With the above guidelines in mind, we note that, with a perfect Bayesian model, 
$$P(C_i|V)=\alpha_i(V) n_i / (\sum_k \alpha_k(V) n_k),
$$
and
\begin{equation}
\label{eq:R-def2}
R(C_i,C'_j)\equiv \frac{1}{n'_j}\sum_{l=1}^{n'_j} P(C_i|z'_lj)  =
n_i \sum_V \frac{\alpha_i(V)\gamma_j(V)}{\sum_k \alpha_k(V) n_k}.
\end{equation}
We can thus define the weights 
$$
\phi(V) =  \frac{1}{\sum_k \alpha_k(V) n_k}
$$ for use in our dot product, and express dot products in terms of
model probabilities,
$$
(\vec{\alpha_i}, \vec{\gamma_j}) = R(C_i,C'_j)/n_i,
$$
$$
(\vec{\alpha_i}, \vec{\alpha_i}) = R(C_i,C_i)/n_i.
$$
This gives us the following generalization for (\ref{eq:ag-cos2}): 
\begin{equation}
\label{eq:ag-cos-gen-2}
f_{ij}^{\rm cosine \ method} =
\frac{1}{\sqrt{n_i}} \cdot 
\frac{ R(C_i,C'_j) }{ R(C_i,C_i)^{1/2} \|\vec{\gamma_j}\|}
\end{equation}
where, however,
$$
\|\vec{\gamma_j}\| \equiv \left(\sum_{v\in \cal{V}}{\gamma_j(v)^2  \phi(v)}\right)^{1/2}
$$ is not expressible in general probability terms. As we did in the
equal-column-size case, we will note that the value
$\|\vec{\gamma_j}\|$ is the same for all elements in the same row of
the confusion matrix. We thus can generalize eq. (\ref{eq:ag-cos3}) as
\begin{equation}
\label{eq:ag-cos-gen-3}
s_{ij} = \frac{N_1}{\sqrt{n_i}} \cdot 
 \frac{ R(C_i,C'_j) }{ R(C_i,C_i)^{1/2} }
\end{equation}
As with (\ref{eq:ag-cos3}), when the matrix of these ratios $s_{ij}$
is reported as the confusion matrix, one can compare values within the
same row of this matrix, but not between rows. 

\section{A symmetric-cosine approach}

{\bf This is presently (2010-12-27) not implemented yet.}

Here we will propose an alternative approach to that outlined in
Sectons \ref{sec:overview}, \ref{sec:agg}. While somewhat ``strange''
in its design, it will generate a confusion matrix with two pleasant
properties:
\begin{enumerate}
\item If the two data sources are identical, the matrix will be symmetric.
\item If the two columns $C_i$ and $C'_j$ are identical, the matrix element $f_ij$ will be equal to 1.
\end{enumerate}

The algorithm (outlined in the general, unequal-column-size, case) is
as follows:

\begin{enumerate}
\item[1] Consider the set of $M_1+M_2$ fields of DS1 and DS2 as a single
  discrimination (set of labels) with $M_1+M_2$ classes (one per field)

\item[2] Create $N_e = \sum_{i=1}^{M_1} n_i + \sum_{i=1}^{M_2} n'_i$
  training examples, each example being the content of one field of
  one record from DS1 or DS2, and carrying the class label based on
  the name of the data set combined with the name of the field in
  question. (When the data are represented with records as rows and
  fields as columns, each "example" introduced at this step will
  correspond to the content of one cell of this table).

\item[3] Tokenize etc. each "example" somehow, converting it into a
  vector in some linear space (a feature vector)

\item[4] Use some kind of Bayesian regression learning algorithm, such
  as one of those implemented by BOXER toolkit learner, on that set of $N_e$
  ``examples'', to come up with a classifier model that
  probabilistically assigns each ``example'' to a ``class'' (i.e., a
  field). 

\item[5] For each pair of columns from DS1+DS2, compute the confusion
  matrix value
\begin{equation}
\label{eq:ag-sym1}
f^{\rm sym} = \sqrt{\frac{ R(C_i|C'_j) R(C'_j|C_i) }{ R(C_i|C_i) R(C'_j|C'_j)}},
\end{equation}
where the averaged probabilities  $R(C_i|C'_j)$ are computed as in (\ref{eq:R-def2}).
\end{enumerate}

It can be shown that the similarity value (\ref{eq:ag-sym1}) is a
cosine of the angle between the vectors $\vec\alpha_i$ and
$\vec\gamma_j$ in the Euclidean space where the dot product is defined
with the weight
$$
\phi(V) =
 \frac{1}{ \sum_{i=1}^{M_1} \alpha_i(V) n_i + \sum_{i=1}^{M_2} \gamma_i(V) n'_i}.
$$

\end{document}
